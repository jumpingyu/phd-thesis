%%
%% Copyright (c) 2018-2019 Weitian LI <liweitianux@sjtu.edu.cn>
%% Creative Commons BY 4.0
%%

\chapter{补充公式}
\label{chap:formulas}

%=====================================================================
\section{结构形成}

在本文所采用的\emph{平直 \lcdm/ 宇宙}中,
\ac{delta-crit}随红移的变化关系可表示为 \cite{kitayama1996,randall2002}:
\begin{equation}
  \label{eq:delta-crit}
  \acs{delta-crit} = \frac{D(z=0)}{D(z)}
    \left[ \frac{3 (12\Cpi)^{2/3}}{20} \right]
    \left[1 + 0.0123 \log_{10} \acs{Om0}(z) \right] ,
\end{equation}
其中 $\acs{Om0}(z)$ 是红移为 $z$ 时的宇宙物质密度参数:
\begin{equation}
  \label{eq:omega-m-z}
  \acs{Om0}(z) = \frac{\acs{Om0} (1+z)^3}{\acs{Om0} (1+z)^3 + \acs{Ol0}} ,
\end{equation}
$D(z)$ 是增长因子 (growth factor),可由下述公式计算
[参见 \citeay{peebles1980}, 式~(13.6)]:
\begin{equation}
  \label{eq:growth-factor}
  D(x) = \frac{(x^3 + 2)^{1/2}}{x^{3/2}}
    \mathlarger{\int_0^x} y^{3/2} (y^3 + 2)^{-3/2} \,\D{y} ,
\end{equation}
并且 $x_0 \equiv (2 \acs{Ol0} \big/ \acs{Om0})^{1/3}$,
$x = x_0 \big/ (1+z)$.

\acl{Hz}为:
\begin{equation}
  \label{eq:hubble-z}
  \acs{Hz} = \acs{H0} \, \acs{Ez}
    = \acs{H0} \sqrt{\acs{Om0} (1+z)^3 + \acs{Ol0}} ,
\end{equation}
其中 \acs{Ez} 是\acl{Ez} \cite{hogg1999}.
该红移处的宇宙临界密度为:
\begin{equation}
  \label{eq:rho-crit}
  \acs{rho-crit}(z) = \frac{3 H^2(z)}{8 \Cpi \acs{G}} ,
\end{equation}
其中 \acs{G} 是\acl{G}.

位于红移 $z$ 处的星系团的\acf{r-vir}定义为该半径范围内星系团的平均密度
是当时宇宙临界密度的 \acs{overdensity-vir} 倍,
由下式给出:
\begin{equation}
  \label{eq:radius-virial}
  \acs{r-vir}(z) = \left[
    \frac{3 \acs{M-vir}}{4\Cpi \acs{overdensity-vir} \acs{rho-crit}(z)}
  \right]^{1/3},
\end{equation}
其中 \acs{M-vir} 是星系团的\acl{M-vir}(通常用作其总引力质量),
\acf{overdensity} \acs{overdensity-vir} 为 \cite{bryan1998}:
\begin{equation}
  \label{eq:delta-vir}
  \acs{overdensity-vir}(z) = 18\Cpi^2 + 82 x - 39 x^2,
\end{equation}
其中 $x \equiv \acs{Om0}(z) - 1$ [见\autoref{eq:omega-m-z}].


%=====================================================================
\section{半径和质量换算}

实际情况中会经常用到星系团的 $r_{200}$ 和 $r_{500}$,分别定义为该半径范围内
星系团的平均密度是当时宇宙临界密度的 200 和 500 倍.
于是,$M_{200}$ 和 $M_{500}$ 分别为 $r_{200}$ 和 $r_{500}$ 之内的总质量.
一般可认为 $r_{200} \simeq \acs{r-vir}$
以及 $r_{500} \simeq 0.65 \, r_{200}$ \cite{ettori2009},
因此 $M_{200} \simeq \acs{M-vir}$,
而 $M_{200}$ 和 $M_{500}$ 之间的换算可参考下述方法.

假定星系团的密度分布符合 \ac{nfw} 模型\cite{navarro1997}:
\begin{equation}
  \label{eq:nfw}
  \rho(r) = \frac{\rho_s}{
    (r / \acs{r-scale}) (1 + r / \acs{r-scale})^2} \,,
\end{equation}
其中 $\rho_s$ 是密度参数,\acs{r-scale} 是\acl{r-scale} (scale radius).
于是半径 $r = s \,\acs{r-vir}$ 之内的总质量可表示为\cite{lokas2001}:
\begin{equation}
  \label{eq:mass-r}
  M(< s \,\acs{r-vir}) = \acs{M-vir}
    \frac{\ln(1 + c s) - c s / (1 + c s)}{\ln(1 + c) - c / (1 + c)} ,
\end{equation}
其中 $c = \acs{r-vir} / \acs{r-scale}$ 为聚集参数 (concentration parameter).
\citeay{duffy2008} 通过数值模拟研究得出该参数与质量存在如下关系:
\begin{equation}
  \label{eq:mass-c}
  c = A \left( \frac{M_{200}}{M_{\R{pivot}}} \right)^B (1+z)^C ,
\end{equation}
其中 $M_{\R{pivot}} = \SI{2e12}{\per\hubble\solarmass}$,
$A = 5.71$, $B = -0.084$, 以及 $C = 0.47$.
利用上述两式便可由 $M_{200}$ 换算得到 $M_{500}$;
反过来并利用迭代法,亦可由 $M_{500}$ 导出 $M_{200}$.


%=====================================================================
\section{时间和距离}

在红移 $z$ 时的\emph{宇宙年龄}具有如下解析计算形式
[参见 \citeay{thomas2000}, 式~(18)]:
\begin{align}
  \label{eq:universe-age}
  t(z; \acs{Om0})
    & = \frac{1}{\acs{H0}} \mathlarger{\int_z^{\infty}}
      \!\frac{\D{z'}}{(1+z')\sqrt{1 + z' (3+3z'+z'^2) \acs{Om0}}}
      \nonumber \\
    & = \frac{2}{3 \acs{H0} \sqrt{1-\acs{Om0}}} \sinh^{-1}
      \!\left( \sqrt{\frac{\Omega_m^{-1} - 1}{(1+z)^3}} \right).
\end{align}

一个物体的\emph{\acf{distance-angular}}
定义为该物体的物理横向尺寸与其对观测者的张角(以 \si{radian} 为单位)之比.
需要注意的是,由于宇宙膨胀的原因,该距离并\emph{不}随红移单调递增,
即相同物理尺寸的物体位于更高红移(如 $z > 1$)处时反而看起来更大 \cite{hogg1999}.

一个物体的\emph{\acf{distance-luminosity}}由以下关系定义:
\begin{equation}
  \label{eq:dl-def}
  \acs{distance-luminosity} \equiv
    \sqrt{\frac{L_{\R{bolo}}}{4\Cpi S_{\R{bolo}}}} ,
\end{equation}
其中 $L_{\R{bolo}}$ 是该物体的本征热光度 (bolometric luminosity),
$S_{\R{bolo}}$ 是测得的热流量 (bolometric flux).
这里的 $L_{\R{bolo}}$ 和 $S_{\R{bolo}}$ 都是对全频段的积分值.

一个位于红移 $z$ 的物体,
其\acl{distance-luminosity}与\acl{distance-angular}之间有以下关系
\cite{weinberg1972,hogg1999,ellis2007}:
\begin{equation}
  \label{eq:dl-da}
  \acs{distance-luminosity}(z) = (1+z)^2 \acs{distance-angular}(z) .
\end{equation}


%% EOF
