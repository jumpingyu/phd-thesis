%%
%% Copyright (c) 2018-2019 Weitian LI <liweitianux@sjtu.edu.cn>
%% Creative Commons BY 4.0
%%

\chapter{低频射电天空的模拟}
\label{chap:simulation}

%=====================================================================
\section{星系团射电晕}
\label{sec:radio-halos}

theoretical studies and models:
turbulent re-acceleration model,
hadronic model (secondary electron models)

FG21sim, ...

Turbulence; Alfven, slow, fast modes; particle acceleration:
Lazarian et al. 2012, SSRv;
Petrosian 2012, SSRv.

%---------------------------------------------------------------------
\subsection{质量函数}

TODO

%---------------------------------------------------------------------
\subsection{并合历史}

TODO

%---------------------------------------------------------------------
\subsection{射电晕的形成与演化}

TODO

%.....................................................................
\subsubsection{热成分性质}

TODO

%.....................................................................
\subsubsection{电子注入过程}

TODO

%.....................................................................
\subsubsection{初始电子能谱}

TODO

%.....................................................................
\subsubsection{湍流加速机制}

TODO

%.....................................................................
\subsubsection{湍流加速时间段}

TODO

%.....................................................................
\subsubsection{能量损失机制}

TODO

%.....................................................................
\subsubsection{数值算法}

TODO

%.....................................................................
\subsubsection{图像生成}

TODO

%.....................................................................
\subsubsection{参数调节}

TODO


%=====================================================================
\section{银河系}

TODO

%---------------------------------------------------------------------
\subsection{同步辐射}

\ac{rad-syn} TODO ...

%---------------------------------------------------------------------
\subsection{自由--自由辐射}
\label{sec:simu-gff}

\ac{rad-ff} ...
\ac{rad-brem} TODO ...


%=====================================================================
\section{河外点源}

TODO


%=====================================================================
\section{再电离信号}

TODO


%=====================================================================
\section{干涉阵列的模拟观测}

TODO

%---------------------------------------------------------------------
\subsection{SKA1-Low~阵列布局}

layout configuration, design goals, descriptions

%---------------------------------------------------------------------
\subsection{模拟观测}

OSKAR simulator

%---------------------------------------------------------------------
\subsection{成像}
\label{sec:imaging}

WSClean imager


%=====================================================================
\section{小结}

TODO


%% EOF
