%%
%% Copyright (c) 2018-2019 Weitian LI <liweitianux@sjtu.edu.cn>
%% Creative Commons BY 4.0
%%

% 中文摘要,约 1000 字
\begin{abstract}

\acf*{eor}是早期宇宙的一段缺乏足够了解的时期,
从宇宙大爆炸之后约 \SI{300}{\Myr} 持续到约 \SI{1}{\Gyr},
对应的红移范围约为 \numrange{6}{15}.
在这个时期,第一代天体刚形成并产生辐射,使中性的重子物质再次被逐渐电离.
研究再电离时期对于理解宇宙早期的结构形成以及星系的形成与演化均有重要意义,
是建立完整的宇宙演化图景的关键环节之一.
在低频射电波段 ($\sim$\,\SIrange{50}{200}{\MHz}) 探测源自再电离时期的
中性氢 \ac*{21cmline}(即 EoR 信号)是目前研究该时期的最直接而有效的办法.
然而,EoR 信号非常微弱,淹没在比其强约 5 个数量级的前景干扰之中.
因此,深刻理解各个前景成分的性质并准确把握它们对 EoR 探测的干扰行为,
是研发具有针对性的前景去除和 EoR 信号分离算法的前提和关键.

银河系的\acs*{rad-syn}和\acs*{rad-ff}以及河外点源是最主要的几种前景成分,
目前已被比较广泛地研究.
除此之外,星系团射电晕作为一类主要的河外展源,将对 EoR 的前景干扰有一定程度的贡献.
在以往的前景研究中,只有很少几个工作简单地包含了射电晕.
这些工作不但对射电晕的模拟过于简单,而且并未进一步分析射电晕对 EoR 信号的具体影响.
因此,本文构建了一个更物理、更完善的模型用来模拟射电晕的低频射电辐射,
然后在考虑干涉阵列的实际仪器效应的情况下,充分评估射电晕对 EoR 信号的影响.

基于 Press--Schechter 理论和\acs*{turbreacc-model},
我们实现了对射电晕的形成和演化过程的完整建模,据此模拟生成了射电晕的天图.
接着,我们采用目前最新的 SKA1-Low 阵列布局开展模拟观测,
得到了整合了实际仪器效应的射电晕图像.
我们在 \numrange{120}{128}、\,\numrange{154}{162}
和 \numrange{192}{200} \si{\MHz} 这三个频带内对比了射电晕和 EoR 信号的功率谱.
一维功率谱的对比显示,在 $\SI{0.1}{\per\Mpc} < k < \SI{2}{\per\Mpc}$ 尺度范围内,
射电晕在三个频带内的功率分别约为 EoR 信号的功率的 \num{e4}、\,\num{e3}
和 \num{e2.5} 倍.
考虑到射电晕的亮度和数目在不同天区里具有显著的涨落,
因此在 68\% 的误差范围内,射电晕的功率能够变化约 \numrange{10}{100} 倍.
在二维功率谱的 EoR 窗口内对功率进行平均,我们发现,
在 $\SI{0.5}{\per\Mpc} \lesssim k \lesssim \SI{1}{\per\Mpc}$ 尺度范围内,
射电晕与 EoR 信号的功率之比在三个频带内的值能够分别达到约
\numrange{230}{800}\%、\,\numrange{18}{95}\% 和 \numrange{7}{40}\%.
说明射电晕泄漏到 EoR 窗口内的功率仍然可能是显著的,
尤其是在 $\sim$\,\SI{120}{\MHz} 的较低频率.
此外,我们还发现仪器的频谱伪结构能够显著增强射电晕在 EoR 窗口内的功率,
远旁瓣里的射电晕也能对 EoR 信号产生严重的污染.
这些结果充分说明了射电晕是一个严重的前景干扰成分,
需要在 EoR 观测中被认真地对待并且有效地扣除.

前景辐射的频谱在本质上非常平滑,而 EoR 信号的频谱则呈锯齿状,所以两者在原则上易于区分.
但是,干涉阵列的波束存在频率依赖效应,这会使前景辐射产生沿频率维度快速变化的起伏.
于是,前景辐射的频谱光滑性受到损坏,由此导致传统\acs*{fg-rm}方法不能有效地分离 EoR 信号.
为了克服复杂的波束效应,我们基于\acs*{dl}方法设计了一个包含 9 个卷积层的\acf*{cdae}
用来分离 EoR 信号.
使用上面模拟得到的 SKA1-Low \enquote{观测}图像来训练 \acs*{cdae} 之后,
我们发现 \acs*{cdae} 重构的 EoR 信号与输入的 EoR 信号之间的\acl*{coef-correlation}%
达到了 $\bar{\ac*{coef-correlation}}_{\R{cdae}} = \num{0.929 +- 0.045}$,
说明 \acs*{cdae} 准确地分离出了 EoR 信号.
相比之下,两种传统的\acs*{fg-rm}方法,即多项式拟合法和\acs*{cwt}法,
分离 EoR 信号的效果分别仅有
$\bar{\ac*{coef-correlation}}_{\R{poly}} = \num{0.296 +- 0.121}$ 和
$\bar{\ac*{coef-correlation}}_{\R{cwt}} = \num{0.198 +- 0.160}$,
说明复杂的波束效应导致两者均无法有效地分离出 EoR 信号.
因此,\acs*{cdae} 能够克服波束效应对前景辐射频谱的影响,从而准确地分离出 EoR 信号.
同时,这些结果也表明\acs*{dl}方法具有在未来 EoR 实验中发挥重要作用的巨大潜力.

\end{abstract}

%---------------------------------------------------------------------

\begin{englishabstract}
\acs*{rh} can impose serious contamination on the EoR detection ...
\end{englishabstract}
