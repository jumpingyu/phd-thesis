%%
%% Copyright (c) 2018-2019 Weitian LI <liweitianux@sjtu.edu.cn>
%% Creative Commons BY 4.0
%%

% 中文摘要,约 1000 字
\begin{abstract}

\acf*{eor}是宇宙演化早期尚不为人所透彻了解的一个重要时期。
它从宇宙大爆炸之后约 3 亿年一直延续到约 10 亿年,
对应的红移范围约为 \numrange{6}{15}。
第一代恒星和星系在这个时期刚形成不久,其紫外和软 X 射线辐射使中性重子物质逐渐被再次电离。
研究 EoR 对于理解第一代天体和宇宙早期结构的形成具有重要意义,
是建立完整的宇宙演化图景的关键之一。
在低频射电波段 ($\sim$\,\SIrange{50}{200}{\MHz}) 探测源自 EoR 的
中性氢 \acs*{21cmline}是目前已提出的研究该时期的最直接和有效的办法。
然而,由于 EoR 信号非常微弱,且淹没在比它强约 5 个数量级的前景干扰之中,
在研究 EoR 时必须深刻理解各个前景干扰成分的性质,
研发具有针对性的\acs*{fg-rm}和 EoR 信号分离算法。

在多种 EoR 前景干扰成分中,
银河系\acs*{rad-syn}和\acs*{rad-ff}、河外点源等几种主要的前景成分,
已被较广泛地研究过。%,(TODO: 简要总结研究程度和结果)。
另一方面,星系团射电晕作为一类较常见的河外射电展源,
也将对 EoR 信号探测产生一定程度的影响。
在以往的前景研究中,只有很少几项工作以较简单的方式触及了射电晕的低频射电辐射。
这些工作对射电晕图像和频谱的模拟过于简化,
而且未进一步分析射电晕辐射对 EoR 信号探测的具体影响。
针对 EoR 探测实验的前景干扰和信号分离难题,本文分两方面开展研究,
首先是对射电晕在低频射电波段的辐射特征进行更完善、更物理的建模,建立更逼真的前景模型,
并在考虑 SKA1-Low 阵列仪器效应的前提下评估射电晕辐射对 EoR 信号探测的影响;
其次是利用改进的前景模型,基于深度学习研发微弱信号分离算法,实现 EoR 信号的准确分离。

基于 Press--Schechter 理论和\acs*{turbreacc-model},
实现了针对射电晕形成和演化过程的完整建模,并据此生成了射电晕的模拟天图,
再利用目前最新的 SKA1-Low 阵列布局模拟了包含仪器效应的 SKA1-Low 射电晕图像。
通过在 \numrange{120}{128}、\numrange{154}{162}
和 \numrange{192}{200} \si{\MHz} 三个频带内比较射电晕和 EoR 信号的一维功率谱,
发现在 $\SI{0.1}{\per\Mpc} < k < \SI{2}{\per\Mpc}$ 尺度范围内
射电晕在三个频带内的典型功率分别约为 EoR 信号的 \num{e4}、\num{e3}
和 \num{e2.5} 倍。
同时通过研究二维功率谱以及 EoR 窗口发现,
在 $\SI{0.5}{\per\Mpc} \lesssim k \lesssim \SI{1}{\per\Mpc}$ 尺度范围
以及 68\% 误差范围内,
射电晕与 EoR 信号的功率比在三个频带内分别可达约
\numrange{230}{800}\%、\numrange{18}{95}\% 和 \numrange{7}{40}\%,
说明射电晕辐射在 EoR 窗口内的泄漏是不可忽略的——
尤其是在 $\sim$\,\SI{120}{\MHz} 的较低频率更为显著。
此外,我们还发现仪器响应所产生的频谱伪结构会显著增强射电晕在 EoR 窗口内的辐射漏泄,
而且旁瓣里的射电晕辐射使这个问题更为严重。
这些结果说明射电晕是一个此前尚未引起充分重视的前景干扰成分,
需要在 EoR 观测中认真地对待。

基于上述改进的前景模型以及模拟的 SKA1-Low 图像,
进一步研究了干涉阵列波束效应对前景辐射频谱光滑性的影响。
干涉阵列波束的频率依赖效应会让原本光滑的前景频谱产生快速变化的起伏,
使前景频谱的光滑性遭到损坏,导致传统\acs*{fg-rm}方法不再适用。
为了解决这个问题,我们基于\acs*{dl}方法设计了一个包含 9 个卷积层的\acf*{cdae}
用来分离 EoR 信号。
使用模拟的 SKA1-Low 图像训练 CDAE 之后,
发现由 CDAE 重建的 EoR 信号与输入的 EoR 信号之间具有\acl*{coef-correlation}
$\bar{\ac*{coef-correlation}}_{\R{cdae}} = \num{0.929 +- 0.045}$,
即 \acs*{cdae} 准确地分离出了 EoR 信号。
与此相比,传统的多项式拟合法和\acs*{cwt}法在分离 EoR 信号时并不成功,仅达到了
$\bar{\ac*{coef-correlation}}_{\R{poly}} = \num{0.296 +- 0.121}$ 和
$\bar{\ac*{coef-correlation}}_{\R{cwt}} = \num{0.198 +- 0.160}$。
因此 CDAE 能够有效克服波束效应对前景辐射频谱光滑性的破坏,
并且准确地分离 EoR 信号,
反映了\acs*{dl}方法在未来 EoR 实验中的潜在重要作用。

\end{abstract}

%---------------------------------------------------------------------

\begin{englishabstract}

The Epoch of Reionization (EoR; $z \sim \numrange{6}{16}$) is
an important stage in the early Universe and is expected to last
from about \SI{300}{\Myr} to \SI{1}{\Gyr} after the Big Bang.
During the EoR, the first stars and galaxies have just formed and
start emitting ultraviolet and soft X-ray photons,
which gradually reionize the neutral baryonic matter.
Studying the EoR is of great value in understanding the first stars
and galaxies and the structure formation of  the early Universe.
Among the methods to probe the EoR, detecting the 21\,cm line of
neutral hydrogen originating from the EoR in the low-frequency
radio band ($\sim$\,\SIrange{50}{200}{\MHz}) is regarded as
the most promising and effective method.
However, the EoR signal is very faint and is buried in the overwhelming
foreground contamination of about 5 orders of magnitude stronger.
Therefore, it is required to thoroughly understand every foreground
component and develop specific foreground removal and EoR signal
separation methods.

Among various foreground components, the Galactic synchrotron and
free--free radiations and the extragalactic point sources are the
major components and have been rather extensively investigated in
the literature.
On the other hand, radio halos in galaxy clusters, which are common
extragalactic extended sources, can also have an impact on the EoR
signal detection.
However, only several works that investigated the EoR foreground
have slightly explored radio halos.
Those works made oversimplifications in simulating the images and
spectra of radio halos and did not further analyze the impacts of
radio halo emission on the EoR signal detection.
In this work, we carry out two studies in regards to the EoR
foreground contamination and signal separation problems.
First, we construct a more complete and physical model for simulating
the radio halo emissions and build a more realistic foreground model.
By taking into account the SKA1-Low instrument effects, we evaluate
the contamination of radio halo emissions on the EoR signal detection.
Secondly, based on the improved foreground model, we develop a new EoR
signal method by utilizing the deep learning algorithms, in order to
achieve accurate separation of the EoR signal.

By employing the Press--Schechter formalism and the turbulent
re-acceleration model, we model the formation and evolution of
radio halos and simulate their sky maps in the low-frequency band.
Then, we adopt the latest SKA1-Low layout configuration and simulate
the SKA1-Low images of radio halos with instrument effects incorporated.
By comparing the one-dimensional power spectra of radio halos and
the EoR signal in the \numrange{120}{128}, \numrange{154}{162},
and \numrange{192}{200} \si{\MHz} frequency bands, we find that
radio halos are generally about  \num{e4}, \num{e3}, and \num{e2.5}
times more luminous than the EoR signal on scales of
$\SI{0.1}{\per\Mpc} < k < \SI{2}{\per\Mpc}$ in the three bands,
respectively.
After examining the two-dimensional power spectra inside appropriately
defined EoR windows, we find that the power leaked by radio halos can
still be significant, as the power ratios of radio halos to the
EoR signal on scales of
$\SI{0.5}{\per\Mpc} \lesssim k \lesssim \SI{1}{\per\Mpc}$ can be up to
about \numrange{230}{800}\%, \numrange{18}{95}\%, and \numrange{7}{40}\%
in the three bands, when the 68\% uncertainties caused by the variation
of the number density of bright radio halos are considered.
Furthermore, we find that frequency artifacts resulted from instrument
response can remarkably increase the power leakage of radio halos in
the EoR window, which becomes more severe due to the radio halos
located inside the side-lobes.
These results show that radio halos are severe foreground sources
and need serious treatments in future EoR experiments.

Based on the improved foreground model and simulated SKA1-Low images
obtained above, we further investigate the beam effects of
interferometers and their impacts on the spectral smoothness of
the foreground emission.
The frequency-dependent beam effects of interferometers will cause
rapid fluctuations along the frequency dimension and damage the
spectral smoothness of the foreground emission, which makes the
traditional foreground removal methods inapplicable.
To address this issue, we propose a deep-learning-based method
that uses a 9-layer convolutional denoising autoencoder (CDAE) to
separate the EoR signal.
After being trained on the SKA1-Low images, the CDAE achieves excellent
performance as the mean correlation coefficient between 
the reconstructed and input EoR signals reaches
$\bar{\ac*{coef-correlation}}_{\R{cdae}} = \num{0.929 +- 0.045}$.
In comparison, the two representative traditional methods, namely the
polynomial fitting method and the continuous wavelet transform method,
both have difficulties in modelling and removing the foreground emission
complicated with the beam effects, yielding only
$\bar{\ac*{coef-correlation}}_{\R{poly}} = \num{0.296 +- 0.121}$ and
$\bar{\ac*{coef-correlation}}_{\R{cwt}} = \num{0.198 +- 0.160}$,
respectively.
In consequence, the CDAE can effectively deal with the spectral
smoothness damage caused by the beam effects and accurately separate
the EoR signal.
This result also exhibits the great potential of deep-learning-based
methods in future EoR experiments.

\end{englishabstract}
