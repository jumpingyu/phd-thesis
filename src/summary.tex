%%
%% Copyright (c) 2019 Weitian LI <liweitianux@sjtu.edu.cn>
%% Creative Commons BY 4.0
%%

\begin{summary}

本文围绕 EoR 探测所面临的前景干扰问题,完成了以下三点工作:
\begin{enumerate}
\item
(a) 为了改进低频射电天空中星系团射电晕的建模,
首先根据扩展 \ac{PS} 理论模拟星系团的并合历史,
然后利用\ac{turbreacc-model}来计算并合所产生的\ac{turbulence}对 \ac{icm}
中的高能电子的再加速过程,
从而实现了对射电晕形成和演化过程的完整建模,
并模拟生成了射电晕的低频射电\ac{skymap}。
另外还模拟了银河系的\ac{rad-syn}和\ac{rad-ff}、河外点源
以及 EoR 信号的低频射电\ac{skymap}。
(b) 采用了目前最新的 SKA1-Low 阵列布局,
模拟了上述各个成分在 \numrange{120}{128}、\numrange{154}{162}
和 \numrange{192}{200} \si{\MHz} 三个频带内的 SKA1-Low 观测图像,
从而将干涉阵列的仪器效应整合到模拟图像和数据分析流程之中。

\item
利用上面获得的 SKA1-Low 观测图像,
计算并对比了射电晕和 EoR 信号在三个频带内的一维和二维\ac{ps},
发现射电晕辐射的功率显著强于待测 EoR 信号。
即使在二维功率谱的 EoR 窗口内,
射电晕所泄漏的污染仍然能够对 EoR 信号的测量产生不可忽略的干扰。
为了更加全面地评估射电晕辐射对 EoR 信号探测的影响,
还考虑了仪器的频谱伪结构以及旁瓣的影响。
这些结果进一步支持了射电晕是一个重要的前景干扰成分,
需要在 EoR 观测中被认真地对待。

\item
利用上述改进的前景模型以及模拟的 SKA1-Low 观测图像,
进一步研究了干涉阵列波束效应对前景辐射频谱光滑性的影响。
基于\ac{dl}方法,设计了一个包括 9 个卷积层的 \ac{cdae} 用来分离 EoR 信号。
使用模拟的 SKA1-Low 观测图像对 \ac{cdae} 进行训练后,
发现 \ac{cdae} 能够准确地分离 EoR 信号,分离效果显著优于传统\ac{fg-rm}方法。
这说明 \ac{cdae} 能够有效地克服波束效应对前景辐射频谱光滑性的破坏,
同时也反映了\ac{dl}方法在未来 EoR 实验中的潜在重要作用。
\end{enumerate}

% TODO...
在前景的高精度模拟方面,后续可开展的工作包括:
改进星系团射电晕的形态结构建构;
模拟和评估星系团的其他弥散射电辐射,比如\ac{rr}和\ac{rmh}。
在 EoR 信号分离方面,后续可将 \ac{cdae} 适用到二维功率谱的处理;
还可以研发\ac{fg-rm}和\ac{fg-avd}的混合方法,尽可能地抑制\ac{fg-wedge}的范围,
扩大 EoR 窗口。

\end{summary}
