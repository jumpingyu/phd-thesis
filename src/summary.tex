%%
%% Copyright (c) 2019 Weitian LI <liweitianux@sjtu.edu.cn>
%% Creative Commons BY 4.0
%%

\begin{summary}

本文围绕 EoR 探测所面临的前景干扰问题,完成了以下四点工作:
\begin{enumerate}
\item
为了改进低频射电天空中星系团射电晕的建模,
我们首先根据扩展 Press--Schechter 理论模拟星系团的并合历史,
然后利用\ac{turbreacc-model}来计算并合所产生的\ac{turbulence}对 \ac{icm}
中的高能电子的再加速过程,
从而实现了对射电晕的形成和演化过程的完整建模,
并模拟生成了射电晕的\ac{skymap}.
另外,我们还模拟了银河系的\ac{rad-syn}和\ac{rad-ff}、河外点源
以及 EoR 信号的\ac{skymap}.

\item
我们采用了目前最新的 SKA1-Low 阵列布局开展模拟观测,
得到上述各个成分在 \numrange{120}{128}、\,\numrange{154}{162}
和 \numrange{192}{200} \si{\MHz} 三个频带内的\enquote{观测}\ac{imgcube}.
这样,干涉阵列实际的仪器效应得以整合到模拟图像之中.

\item
利用上面获得的 SKA1-Low \enquote{观测}\ac{imgcube},
我们计算并对比了射电晕和 EoR 信号在三个频带内的一维和二维\ac{ps}.
对比结果显示了射电晕的功率显著强于 EoR 信号的功率.
即使在二维功率谱的 EoR 窗口内,射电晕所泄漏的污染仍然能够对
EoR 信号的准确测量产生不可忽略的干扰.
为了更加全面地评估射电晕对 EoR 探测的影响,
我们还考虑了仪器的频谱伪结构以及远旁瓣的影响.
这些结果进一步支持了射电晕是一个重要的前景干扰成分,
需要在 EoR 观测中被认真地对待和有效地扣除.

\item
基于\ac{dl}方法,我们设计了一个包括 9 个卷积层的 \ac{cdae} 用来分离 EoR 信号.
使用上面模拟的\enquote{观测}\ac{imgcube},我们训练了 \ac{cdae} 并演示其性能.
结果显示,训练好的 \ac{cdae} 能够准确地分离 EoR 信号,
分离效果显著优于传统\ac{fg-rm}方法.
这说明 \ac{cdae} 成功地克服了干涉阵列的波束效应对前景辐射频谱的影响,
同时也展示了\ac{dl}方法的巨大潜力,将在未来 EoR 实验中发挥重要作用.
\end{enumerate}

在前景的高精度模拟方面,后续可开展的工作包括:
改进星系团射电晕的形态结构建构;
模拟和评估星系团的其他弥散射电辐射,比如\ac{rr}和\ac{rmh}.
在 EoR 信号分离方面,后续可将 \ac{cdae} 适用到二维功率谱的处理;
还可以研发\ac{fg-rm}和\ac{fg-avd}的混合方法,尽可能地抑制\ac{fg-wedge}的范围,
扩大 EoR 窗口.

\end{summary}
