%%
%% Copyright (c) 2019 Weitian LI <liweitianux@sjtu.edu.cn>
%% Creative Commons BY 4.0
%%

\begin{summary}

在低频射电波段探测 EoR 信号是目前研究该时期最直接和有效的办法,
但是 EoR 探测面临诸多挑战,其中一个关键困难便是强烈的前景干扰。
本文借助 SKA1-Low 干涉阵列,围绕 EoR 探测所面临的前景干扰问题,完成了以下三点工作:
\begin{enumerate}
\item
(a) 为了改进低频射电天空中星系团射电晕的建模,
首先根据扩展 \ac{PS} 理论模拟星系团的并合历史,
然后利用\ac{turbreacc-model}来计算并合所产生的\ac{turbulence}对 \ac{icm}
中的高能电子的再加速过程,
从而实现了对射电晕形成和演化过程的完整建模,
并模拟生成了射电晕的低频射电\ac{skymap}。
另外还模拟了银河系的\ac{rad-syn}和\ac{rad-ff}、河外点源
以及 EoR 信号的低频射电\ac{skymap}。
(b) 采用目前最新的 SKA1-Low 阵列布局,
模拟了上述各个成分在 \numrange{120}{128}、\numrange{154}{162}
和 \numrange{192}{200} \si{\MHz} 三个频带内的 SKA1-Low 观测图像,
从而将干涉阵列的仪器效应整合到模拟图像和数据分析流程之中。

\item
利用上面获得的 SKA1-Low 观测图像,
计算并对比了射电晕和 EoR 信号在三个频带内的一维和二维\ac{ps},
发现射电晕辐射的功率显著强于待测 EoR 信号。
即使在二维功率谱的 EoR 窗口内,
射电晕所泄漏的污染仍然能够对 EoR 信号的测量产生不可忽略的干扰。
为了更加全面地评估射电晕辐射对 EoR 信号探测的影响,
还考虑了仪器的频谱伪结构以及旁瓣的影响。
这些结果进一步支持了射电晕是一个重要的前景干扰成分,
需要在 EoR 观测中认真对待。

\item
利用上述改进的前景模型以及模拟的 SKA1-Low 观测图像,
进一步研究了干涉阵列的波束效应对前景辐射频谱光滑性的影响。
基于\ac{dl}方法,设计了一个包括 9 个卷积层的 \ac{cdae} 用来分离 EoR 信号。
使用模拟的 SKA1-Low 观测图像对 \ac{cdae} 进行训练后,
发现 \ac{cdae} 能够准确地分离 EoR 信号,分离效果显著优于传统\ac{fg-rm}方法。
这说明 \ac{cdae} 能够有效地克服波束效应对前景辐射频谱光滑性的破坏,
同时也反映了\ac{dl}方法在未来 EoR 实验中的潜在重要作用。
\end{enumerate}

%---------------------------------------------------------------------

随着 \ac{mwa} 二期 \cite{wayth2018} 升级完成并开展观测,
以及 SKA1-Low 开始加速建设,解决 EoR 探测的前景干扰问题的需求也越来越迫切。
基于在本工作中积累的技术和经验,我们认为后续可开展的工作主要有:
\begin{itemize}
\item 前景辐射建模的改进:
  \begin{itemize}
    \item 改进射电晕形态结构的模拟,生成形态更逼真的射电晕图像,
      比如利用\ac{vae}\cite{kingma2013} 或\ac{gan}\cite{goodfellow2014};
    \item 增加对星系团其他弥散射电辐射的模拟,比如\ac{rr}和\ac{rmh};
    \item 将流体动力学模拟与本文构建的射电晕模型结合起来,
      比如先通过流体动力学模拟(甚至宇宙学模拟)获得星系团的成长过程,
      然后利用本文的射电晕模型计算射电辐射;
    \item 使用\ac{delay-spec}方法\cite{parsons2012}来计算\ac{ps},
      不经过成像步骤,然后评估前景辐射对 EoR 信号探测的干扰情况;
    \item 深入挖掘低频射电巡天数据,主要包括:
      \ac{mwa} 的 \ac{gleam}\cite{wayth2015,hurleyWalker2017}
      和 \ac{gleam-x}\cite{hurleyWalker2017prop},
      \ac{lofar} 的 \ac{lotss}\cite{shimwell2017,shimwell2019},
      \ac{gmrt} 的 \ac{tgss}\cite{intema2017}。
  \end{itemize}

\item EoR 信号分离算法的研发:
  \begin{itemize}
    \item 将本文设计的 \ac{cdae} 适用到二维\ac{ps}的处理,
      因为二维\ac{ps}是目前广泛使用的 EoR 探测方法,更接近实际应用;
    \item 除了 \ac{cdae},尝试将其他\ac{dl}算法用于 EoR 信号分离问题,
      比如\ac{res-nn} \cite{he2016};
    \item 图像的相邻像素点存在一定的联系(如属于同一个源),
      研发能够利用图像的空间关联信息的 EoR 信号分离算法;
    \item 研发\ac{fg-rm}和\ac{fg-avd}的混合方法 \cite{kerrigan2018},
      能够尽可能地抑制\ac{fg-wedge}的范围,扩大 EoR 窗口。
  \end{itemize}
\end{itemize}

\end{summary}
