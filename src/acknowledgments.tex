%%
%% Copyright (c) 2018-2019 Weitian LI <liweitianux@sjtu.edu.cn>
%% Creative Commons BY 4.0
%%

\begin{thanks}

首先,衷心感谢导师徐海光教授的培养,
不仅教导我在研究、做事、写作等方面精益求精,还培养我为人处世的能力,
让我获得做学问和做人的全面成长。

感谢本校、国家天文台、上海天文台的各位老师长期以来的教诲和关心。
特别感谢
武向平院士、
陈列文教授、沈俊太教授、王挺贵教授、张骏教授、张鹏杰教授、郑茂俊教授、朱杰教授、朱卡的教授、
刘成则老师
在学位论文的评阅和答辩过程中的批评和指导。
还要感谢物理与天文学院的领导、老师、行政人员的关心和帮助。

感谢师兄、师姐的帮助,特别是王婧颖师姐和顾俊骅师兄。
感谢课题组里一起奋斗的小伙伴:
胡丹、连晓丽、刘宇星、马志贤、单晨曦、郑东超、朱永凯、朱正浩。
还要感谢 Jeffrey Hsu 以及好友朱睿敏对论文的帮助。

本工作使用了
由 Fred Dulwich 开发的
\href{https://github.com/OxfordSKA/OSKAR}{\texttt{OSKAR}} 模拟软件
以及他提供的最新 SKA1-Low 阵列布局、
由 André Offringa 开发的
\href{https://sourceforge.net/projects/wsclean/}{\texttt{WSClean}} 成像软件、
由 Mathieu Remazeilles 提供的高分辨率 Haslam \SI{408}{\MHz} 银河系全天辐射图、
由 Giovanna Giardino 提供的银河系同步辐射谱指数的全天图、
由 \href{http://homepage.sns.it/mesinger/EOS.html}{%
  \textit{Evolution Of 21\,cm Structure}} 项目公开的 EoR 模拟数据。
本论文还使用了由%
\href{http://sjtug.org/}{上海交通大学 *nix 用户组}维护的
\href{https://github.com/sjtug/SJTUThesis}{\XeLaTeX{} 学位论文模板}。
在此非常感谢他们。

感谢中国科学技术部(项目编号:2018YFA0404601、2017YFF0210903)
和国家自然科学基金委(项目编号:11433002、11621303、11835009、61371147、11125313)
的资助。

感谢学校提供的优良互联网条件,让本工作能够顺利开展。
同时,我也将无法忘记在这里结识的一群好朋友。

本工作的完成离不开以下项目/工具/网站的支持:
\href{https://arxiv.org/}{arXiv},
\href{http://ads.harvard.edu/}{Astrophysics Data System (ADS)},
\href{https://www.debian.org/}{Debian GNU/Linux},
\href{https://www.draw.io/}{Draw.io},
\href{https://www.gimp.org/}{GIMP},
\href{https://git-scm.com/}{Git},
\href{https://github.com/}{Github},
\href{https://www.google.com/}{Google Search},
\href{https://imagemagick.org/}{ImageMagick},
\href{https://keras.io/}{Keras},
\href{https://www.latex-project.org/}{\LaTeX} (%
\href{https://github.com/josephwright/beamer}{Beamer},
\href{https://www.tug.org/texlive/}{\TeX{} Live}),
\href{https://www.libreoffice.org/}{LibreOffice},
\href{https://www.mozilla.org/en-US/firefox/}{Mozilla Firefox},
\href{https://ned.ipac.caltech.edu/}{NASA/IPAC Extragalactic Database (NED)},
\href{https://developer.nvidia.com/cuda-zone}{NVIDIA CUDA},
\href{https://okular.kde.org/}{Okular},
\href{https://www.openssh.com/}{OpenSSH},
\href{https://www.python.org/}{Python} (%
\href{https://www.astropy.org/}{Astropy},
\href{https://jupyter.org/}{Jupyter},
\href{https://matplotlib.org/}{Matplotlib},
\href{https://www.numpy.org/}{NumPy},
\href{https://pandas.pydata.org/}{Pandas},
\href{https://scipy.org/}{SciPy}),
\href{http://jonls.dk/redshift/}{Redshift},
\href{https://rsync.samba.org/}{Rsync},
\href{http://ds9.si.edu/}{SAOImage DS9},
\href{https://sci-hub.tw/}{Sci-Hub},
\href{https://shadowsocks.org/}{ShadowSocks},
\href{http://simbad.u-strasbg.fr/simbad/}{SIMBAD Astronomical Database},
\href{https://stackoverflow.com/}{Stack Overflow},
\href{https://syncthing.net/}{Syncthing},
\href{https://github.com/tmux/tmux}{Tmux},
\href{https://www.vim.org/}{Vim},
\href{https://www.wechat.com/}{WeChat},
\href{https://www.wikipedia.org/}{Wikipedia},
\href{http://wps-community.org/}{WPS Office},
\href{https://www.xfce.org/}{Xfce},
\href{http://www.zsh.org/}{Zsh}。

此外,感谢 \href{https://www.dragonflybsd.org/}{DragonFly BSD}
项目及其 IRC 上的朋友,特别是乔彦民 (\texttt{sephe})、
Sascha Wildner (\texttt{swildner})、
Matthew Dillon (\texttt{dillon})、
Antonio Huete Jimenez (\texttt{tuxillo})、
Rimvydas Jasinskas (\texttt{zrj})。

最后,感谢父母和亲人的无私关爱和支持。
还要感谢女友尹璐璐坚定不移的支持和鼓励,与我携手走过这段难忘岁月。

\end{thanks}
