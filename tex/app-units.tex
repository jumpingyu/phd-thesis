%%
%% Copyright (c) 2019 Weitian LI <liweitianux@sjtu.edu.cn>
%% Creative Commons BY 4.0
%%

\chapter{单位换算}
\label{chap:units}

在 \ac{cgs} 单位制中,力 (force) $F$ 的单位及其换算关系为:
\begin{align}
  \label{eq:cgs-force}
  \SI{1}{\dyne}
    & = \SI{1}{\gram \cm \second\tothe{-2}} \\
    & = \SI{e-5}{\newton} .
\end{align}
能量 (energy) $E$ 的单位及其换算关系为:
\begin{align}
  \label{eq:cgs-work}
  \SI{1}{\erg}
    & = \SI{1}{\gram \cm\tothe{2} \second\tothe{-2}} \\
    & = \SI{e-7}{\joule} .
\end{align}

针对电磁学,有多个基于 \ac{cgs} 单位制的扩展,其中\ac{g-units}是最常用的。
在该单位制中,电荷 $q$ 的单位及其换算关系为:
\begin{align}
  \label{eq:gu-electrical-charge}
  \SI{1}{\esu}
    & \equiv \SI{1}{\statcoulomb} \equiv \SI{1}{\franklin} \\
    & = \SI{1}{\dyne\tothe{1/2} \cm} \\
    & = \SI{1}{\cm\tothe{3/2} \gram\tothe{1/2} \per\second} \\
    & = (10 c)^{-1}\,\si{\coulomb} \approx \SI{3.33564e-10}{\coulomb} ,
\end{align}
其中 $c = \SI{299792458}{\meter\per\second} $ 为真空中的光速。
磁感应强度 (magnetic induction) $\B{B}$ 的单位及其换算关系为:
\begin{align}
  \label{eq:gu-b-field}
  \SI{1}{\gauss}
    & = \SI{1}{\esu \cm\tothe{-2}} \\
    & = \SI{1}{\cm\tothe{-1/2} \gram\tothe{1/2} \second} \\
    & = \SI{e-4}{\tesla} .
\end{align}

天文中常用的流量密度 (flux density) $S$ 的单位及其换算关系为:
\begin{align}
  \label{eq:jy-conv}
  \SI{1}{\jansky}
    & = \SI{e-26}{\watt \meter\tothe{-2} \per\hertz} \\
    & = \SI{e-23}{\erg \per\second \cm\tothe{-2} \per\hertz} .
\end{align}


%% EOF
