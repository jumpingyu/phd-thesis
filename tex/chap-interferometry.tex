%%
%% Copyright (c) 2018 Weitian LI <liweitianux@sjtu.edu.cn>
%% Creative Commons BY 4.0
%%

\chapter{射电干涉技术基础}
\label{chap:interferometry}

%=====================================================================
\section{射电天文学简介}
\label{sec:radio-astronomy}

TODO

%---------------------------------------------------------------------
\subsection{射电天文学是什么?}

TODO

%---------------------------------------------------------------------
\subsection{射电窗口}

TODO

%---------------------------------------------------------------------
\subsection{机遇和挑战}

TODO


%=====================================================================
\section{辐射理论}
\label{sec:radiation}

TODO

%---------------------------------------------------------------------
\subsection{亮度和流量密度}

TODO

%---------------------------------------------------------------------
\subsection{黑体辐射和亮温度}

TODO


%=====================================================================
\section{天线原理}
\label{sec:antenna}

TODO

%---------------------------------------------------------------------
\subsection{辐射方向图}

TODO

%---------------------------------------------------------------------
\subsection{增益和阻抗}

TODO

%---------------------------------------------------------------------
\subsection{主瓣和旁瓣}

TODO

ERA: eq:(3.96,3.118)

%---------------------------------------------------------------------
\subsection{有效面积}

TODO

%---------------------------------------------------------------------
\subsection{互易定理}

(???) TODO

%---------------------------------------------------------------------
\subsection{天线温度}

TODO


%=====================================================================
\section{干涉仪和综合孔径}
\label{sec:interferometer}

TODO

%---------------------------------------------------------------------
\subsection{基本原理}

TODO

二元干涉仪

%---------------------------------------------------------------------
\subsection{综合孔径}

TODO

坐标系统, 优点和缺点

%---------------------------------------------------------------------
\subsection{可视度}

TODO

%---------------------------------------------------------------------
\subsection{\texorpdfstring{$uv$}{uv} 覆盖}

TODO

integration time, earth rotation, phase tracking

%---------------------------------------------------------------------
\subsection{三个波束}

antenna beam, (??? primary beam), station beam, sythesized beam (PSF)

%---------------------------------------------------------------------
\subsection{三个中心}

phase center, pointing center, delay center

phase tracking, drift scan

%---------------------------------------------------------------------
\subsection{灵敏度}

point-source sensitivity, brightness sensitivity

ERA: 3.6.3.2:confusion

%---------------------------------------------------------------------
\subsection{数字波束合成}

multi-beam, phased array, 优点和缺点

%---------------------------------------------------------------------
\subsection{脏图}

weighting (natural, uniform, robust/Briggs)

%---------------------------------------------------------------------
\subsection{CLEAN 算法}

TODO

%---------------------------------------------------------------------
\subsection{大视场成像}

$w$-term, $w$-projection, $w$-stacking


%=====================================================================
\section{低频干涉阵列}
\label{sec:instruments}

TODO

%---------------------------------------------------------------------
\subsection{21CMA}

TODO

%---------------------------------------------------------------------
\subsection{LOFAR}

TODO

%---------------------------------------------------------------------
\subsection{MWA}

TODO

%---------------------------------------------------------------------
\subsection{SKA}

TODO

%---------------------------------------------------------------------
\subsection{HERA}

TODO

%---------------------------------------------------------------------
\subsection{MITEoR}

TODO

%---------------------------------------------------------------------
\subsection{LWA}

TODO


%=====================================================================
\section{小结}

TODO


%% EOF
