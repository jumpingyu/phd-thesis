%%
%% Copyright (c) 2018 Weitian LI <liweitianux@sjtu.edu.cn>
%% Creative Commons BY 4.0
%%

\DeclareAcronym{Dz}{
  short = \ensuremath{D(z)},
  long = 增长因子,
  foreign = growth factor,
  class = symbol,
}

\DeclareAcronym{delta-crit}{
  short = \ensuremath{\delta_c(z)},
  long = 临界线性过密度,
  foreign = critical linear overdensity,
  class = symbol,
}

\DeclareAcronym{Delta-vir}{
  short = \ensuremath{\Delta_{\R{vir}}(z)},
  long = 星系团的平均过密度,
  foreign = average overdensity,
  class = symbol,
}

\DeclareAcronym{Ez}{
  short = \ensuremath{E(z)},
  long = 红移演化因子,
  class = symbol,
}

\DeclareAcronym{G}{
  short = \ensuremath{G},
  long = 引力常数,
  class = symbol,
}

\DeclareAcronym{h}{
  short = \si{\hubble},
  long = 无量纲 Hubble 常数,
  class = symbol,
}

\DeclareAcronym{H0}{
  short = \ensuremath{H_0},
  long = 当前的 Hubble 常数,
  class = symbol,
}

\DeclareAcronym{Hz}{
  short = \ensuremath{H(z)},
  long = 红移为 $z$ 时的 Hubble 常数,
  class = symbol,
}

\DeclareAcronym{M-vir}{
  short = \ensuremath{M_{\R{vir}}},
  long = 维里质量,
  foreign = virial mass,
  class = symbol,
}

\DeclareAcronym{ns}{
  short = \ensuremath{n_s},
  long = 原初扰动的标量谱指数,
  foreign = scalar spectral index,
  class = symbol,
}

\DeclareAcronym{Ob0}{
  short = \ensuremath{\Omega_b},
  long = 当前的宇宙重子物质密度参数,
  sort = Omega-b,
  class = symbol,
}

\DeclareAcronym{Ofz}{
  short = \ensuremath{\Omega_f(z)},
  long = 红移为 $z$ 时的宇宙物质比例,
  sort = Omega-f-z,
  class = symbol,
}

\DeclareAcronym{Om0}{
  short = \ensuremath{\Omega_m},
  long = 当前的宇宙物质(包括重子物质和暗物质)密度参数,
  sort = Omega-m,
  class = symbol,
}

\DeclareAcronym{Ol0}{
  short = \ensuremath{\Omega_{\Lambda}},
  long = 当前的宇宙常数或真空能密度参数,
  sort = Omega-Lambda,
  class = symbol,
}

\DeclareAcronym{r-vir}{
  short = \ensuremath{r_{\R{vir}}},
  long = 维里半径,
  foreign = virial radius,
  class = symbol,
}

\DeclareAcronym{rho-crit}{
  short = \ensuremath{\rho_{\R{crit}}(z)},
  long = 红移为 $z$ 时的宇宙临界密度,
  foreign = critical density,
  class = symbol,
}

\DeclareAcronym{sigma8}{
  short = \ensuremath{\sigma_8},
  long = 原初扰动在 \SI{8}{\per\hubble\Mpc} 尺度上的幅度,
  class = symbol,
}

\endinput
