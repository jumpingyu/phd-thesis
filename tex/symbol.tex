%%
%% Copyright (c) 2018 Weitian LI <liweitianux@sjtu.edu.cn>
%% Creative Commons BY 4.0
%%

\DeclareAcronym{c}{
  short = \ensuremath{c},
  long = 光速,
  class = symbol,
}

\DeclareAcronym{cs}{
  short = \ensuremath{c_s},
  long = 声速,
  class = symbol,
}

\DeclareAcronym{DA}{
  short = \ensuremath{D_{\!A}},
  long = 角直径距离,
  foreign = angular diameter distance,
  class = symbol,
}

\DeclareAcronym{DL}{
  short = \ensuremath{D_{\!L}},
  long = 光度距离,
  foreign = luminosity distance,
  class = symbol,
}

\DeclareAcronym{DM}{
  short = \ensuremath{D_{\!M}},
  long = 横向共动距离,
  foreign = transverse comoving distance,
  class = symbol,
}

\DeclareAcronym{Dz}{
  short = \ensuremath{D(z)},
  long = 增长因子,
  foreign = growth factor,
  class = symbol,
}

\DeclareAcronym{delta-crit}{
  short = \ensuremath{\delta_c(z)},
  long = 临界线性过密度,
  foreign = critical linear overdensity,
  class = symbol,
}

\DeclareAcronym{Delta-vir}{
  short = \ensuremath{\Delta_{\R{vir}}(z)},
  long = 星系团的平均过密度,
  foreign = average overdensity,
  class = symbol,
}

\DeclareAcronym{Ez}{
  short = \ensuremath{E(z)},
  long = 红移演化因子,
  class = symbol,
}

\DeclareAcronym{freq}{
  short = \ensuremath{\nu},
  long = 频率,
  sort = nu,
  class = symbol,
}

\DeclareAcronym{G}{
  short = \ensuremath{G},
  long = 引力常数,
  class = symbol,
}

\DeclareAcronym{h}{
  short = \si{\hubble},
  long = 无量纲 Hubble 常数,
  class = symbol,
}

\DeclareAcronym{H0}{
  short = \ensuremath{H_0},
  long = 当前的 Hubble 常数,
  class = symbol,
}

\DeclareAcronym{Hz}{
  short = \ensuremath{H(z)},
  long = 红移为 $z$ 时的 Hubble 常数,
  class = symbol,
}

\DeclareAcronym{Ifreq}{
  short = \ensuremath{I_{\nu}},
  long = 频率 $\nu$ 处的表面亮度,
  class = symbol,
}

\DeclareAcronym{L-bolo}{
  short = \ensuremath{L_{\R{bolo}}},
  long = 热光度,
  foreign = bolometric luminosity,
  class = symbol,
}

\DeclareAcronym{M-vir}{
  short = \ensuremath{M_{\R{vir}}},
  long = 维里质量,
  foreign = virial mass,
  class = symbol,
}

\DeclareAcronym{N-ant}{
  short = \ensuremath{N_{\!A}},
  long = 天线数目,
  class = symbol,
}

\DeclareAcronym{ns}{
  short = \ensuremath{n_s},
  long = 原初扰动的标量谱指数,
  foreign = scalar spectral index,
  class = symbol,
}

\DeclareAcronym{Ob0}{
  short = \ensuremath{\Omega_b},
  long = 当前的宇宙重子物质密度参数,
  sort = Omega-b,
  class = symbol,
}

\DeclareAcronym{Ofz}{
  short = \ensuremath{\Omega_f(z)},
  long = 红移为 $z$ 时的宇宙物质比例,
  sort = Omega-f-z,
  class = symbol,
}

\DeclareAcronym{Om0}{
  short = \ensuremath{\Omega_m},
  long = 当前的宇宙物质(包括重子物质和暗物质)密度参数,
  sort = Omega-m,
  class = symbol,
}

\DeclareAcronym{Ol0}{
  short = \ensuremath{\Omega_{\Lambda}},
  long = 当前的宇宙常数或真空能密度参数,
  sort = Omega-Lambda,
  class = symbol,
}

\DeclareAcronym{r-vir}{
  short = \ensuremath{r_{\R{vir}}},
  long = 维里半径,
  foreign = virial radius,
  class = symbol,
}

\DeclareAcronym{rho-crit}{
  short = \ensuremath{\rho_{\R{crit}}(z)},
  long = 红移为 $z$ 时的宇宙临界密度,
  foreign = critical density,
  class = symbol,
}

\DeclareAcronym{S-bolo}{
  short = \ensuremath{S_{\R{bolo}}},
  long = 热流量,
  foreign = bolometric flux,
  class = symbol,
}

\DeclareAcronym{S-uv}{
  short = \ensuremath{S(u,v)},
  long = 采样函数,
  foreign = sampling function,
  class = symbol,
}

\DeclareAcronym{sb-width}{
  short = \ensuremath{\theta_s},
  long = 综合波束宽度,
  foreign = synthesized beamwidth,
  sort = theta-s,
  class = symbol,
}

\DeclareAcronym{Sfreq}{
  short = \ensuremath{S(\nu)},
  long = 频率 $\nu$ 处的流量密度,
  foreign = flux density,
  sort = S-nu,
  class = symbol,
}

\DeclareAcronym{Sidx}{
  short = \ensuremath{\alpha},
  long = 谱指数,
  foreign = spectral index,
  sort = alpha,
  class = symbol,
}

\DeclareAcronym{sigma8}{
  short = \ensuremath{\sigma_8},
  long = 原初扰动在 \SI{8}{\per\hubble\Mpc} 尺度上的幅度,
  class = symbol,
}

\DeclareAcronym{tau-c}{
  short = \ensuremath{\tau_0},
  long = 补偿延迟,
  foreign = compensating delay,
  class = symbol,
}

\DeclareAcronym{tau-g}{
  short = \ensuremath{\tau_g},
  long = 几何延迟,
  foreign = geometric delay,
  class = symbol,
}

\DeclareAcronym{Vis}{
  short = \ensuremath{\symscr{V}},
  long = 复可见度,
  foreign = complex visibility,
  class = symbol,
}

\DeclareAcronym{z}{
  short = \ensuremath{z},
  long = 红移,
  class = symbol,
}

\endinput
