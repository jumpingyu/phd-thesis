%%
%% Copyright (c) 2018 Weitian LI <liweitianux@sjtu.edu.cn>
%% Creative Commons BY 4.0
%%

\chapter{宇宙再电离时期的探测}
\label{chap:detection}


%=====================================================================
\section{早期宇宙}
\label{sec:early-universe}

TODO


%=====================================================================
\section{中性氢 \SI{21}{\cm} 信号}
\label{sec:21cm-signal}

TODO


%=====================================================================
\section{探测方法}

TODO

%---------------------------------------------------------------------
\subsection{直接成像法}

TODO

%---------------------------------------------------------------------
\subsection{功率谱测量法}

TODO

%---------------------------------------------------------------------
\subsection{主要困难}

cosmological foreground contamination,
instrumental effects, ionopsheric distortion,
radio frequency interference (RFI),
calibration imperfection, etc.


%=====================================================================
\section{主要前景成分}

TODO

%---------------------------------------------------------------------
\subsection{银河系\acl{synrad}}  % 同步辐射

TODO

%---------------------------------------------------------------------
\subsection{银河系\acl{ffrad}}  % 自由-自由辐射

TODO

%---------------------------------------------------------------------
\subsection{河外\acl{pntsrc}}  % 河外点源

TODO

%---------------------------------------------------------------------
\subsection{河外\acl{extsrc}}  % 河外展源

galaxy clusters (halos, relics, mini-halos),
intergalactic medium (virial shocks),
cosmic filaments, etc.

星系团射电晕


%=====================================================================
\section{前景处理方法}

key characteristic: frequency structure difference between
the 21~cm signal and foreground emission.

%---------------------------------------------------------------------
\subsection{\acl{fgrm}}  % 前景扣除法

TODO

%---------------------------------------------------------------------
\subsection{\acl{fgav}}  % 前景回避法

2D power spectrum, EoR window


%=====================================================================
\section{\acl{eor-window}}  % 再电离窗口
\label{sec:eor-window}

EoR window, foreground wedge, explanation ...


%=====================================================================
\section{小结}

TODO


%% EOF
