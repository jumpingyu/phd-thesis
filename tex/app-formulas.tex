%%
%% Copyright (c) 2018 Weitian LI <liweitianux@sjtu.edu.cn>
%% Creative Commons BY 4.0
%%

\chapter{补充公式}
\label{chap:formulas}


在本文所采用的平直 \lcdm/ 宇宙中,
\ac{delta-crit}随红移的变化关系可表示为 \cite{kitayama1996,randall2002}:
\begin{equation}
  \label{eq:delta-crit}
  \acs{delta-crit} = \frac{D(z=0)}{\acs{Dz}}
    \left[ \frac{3 (12\pi)^{2/3}}{20} \right]
    \left[1 + 0.0123 \log_{10} \acs{Ofz} \right] ,
\end{equation}
其中 \acs{Ofz} 是\acl{Ofz}:
\begin{equation}
  \label{eq:omega-fz}
  \acs{Ofz} = \frac{\acs{Om0} (1+z)^3}{\acs{Om0} (1+z)^3 + \acs{Ol0}} ,
\end{equation}
\acs{Dz} 是\acl{Dz},可由下述公式计算
[参见 \textcite{peebles1980}, 式~(13.6)]:
\begin{equation}
  \label{eq:growth-factor}
  D(x) = \frac{(x^3 + 2)^{1/2}}{x^{3/2}}
    \mathlarger{\int_0^x} y^{3/2} (y^3 + 2)^{-3/2} \,\D{y} ,
\end{equation}
并且 $x_0 \equiv (2 \acs{Ol0}/\acs{Om0})^{1/3}$、$x = x_0 / (1+z)$。

在红移 $z$ 时的宇宙年龄具有如下解析计算形式
[参见 \textcite{thomas2000}, 式~(18)]:
\begin{align}
  \label{eq:universe-age}
  t(z; \acs{Om0})
    & = \frac{1}{\acs{H0}} \mathlarger{\int_z^{\infty}}
      \!\frac{\D{z'}}{(1+z')\sqrt{1 + z' (3+3z'+z'^2) \acs{Om0}}}
      \nonumber \\
    & = \frac{2}{3 \acs{H0} \sqrt{1-\acs{Om0}}} \sinh^{-1}
      \!\left( \sqrt{\frac{\Omega_m^{-1} - 1}{(1+z)^3}} \right).
\end{align}

\acl{Hz}为:
\begin{equation}
  \label{eq:hubble-z}
  \acs{Hz} = \acs{H0} \, \acs{Ez}
    = \acs{H0} \sqrt{\acs{Om0} (1+z)^3 + \acs{Ol0}} ,
\end{equation}
其中 \acs{Ez} 是\acl{Ez} \cite{hogg1999}。
该红移处的宇宙临界密度为:
\begin{equation}
  \label{eq:rho-crit}
  \acs{rho-crit} = \frac{3 H^2(z)}{8 \pi \acs{G}} ,
\end{equation}
其中 \acs{G} 是\acl{G}。

星系团的\acf{r-vir}由下式给出:
\begin{equation}
  \label{eq:radius-virial}
  \acs{r-vir} = \left[
    \frac{3 \acs{M-vir}}{4\pi \acs{Delta-vir} \acs{rho-crit}}
  \right]^{1/3},
\end{equation}
其中 \acs{M-vir} 是星系团的\acl{M-vir}(亦可当作其总质量)、
\acs{Delta-vir} 是\acl{Delta-vir},由下式给出
\cite{kitayama1996,cassano2005}:
\begin{equation}
  \label{eq:delta-vir}
  \acs{Delta-vir} = 18\pi^2 \left[ 1 + 0.4093 \, w(z)^{0.9052} \right],
\end{equation}
并且 $w(z) \equiv \Omega_f^{-1}(z) - 1$。

一个物体的\acf{DA}定义为该物体的物理横向尺寸与其对观测者的张角
(以 \si{radian} 为单位)之比。
需要注意的是,由于宇宙膨胀的原因,该距离并不随红移单调递增,
即相同物理尺寸的物体位于更高红移(如 $z > 1$)处时反而看起来更大 \cite{hogg1999}。

一个物体的\acf{DL}由以下关系定义:
\begin{equation}
  \label{eq:dl-def}
  \acs{DL} \equiv \sqrt{\frac{\acs{L-bolo}}{4\pi \acs{S-bolo}}},
\end{equation}
其中 \acs{L-bolo} 是该物体的本征\acl{L-bolo},
\acs{S-bolo} 是测得的\acl{S-bolo}(对全频段积分)。

对于一个位于红移 $z$ 的物体,其\acl{DL}与\acl{DA}之间存在以下关联
\cite{weinberg1972,hogg1999,ellis2007}:
\begin{equation}
  \label{eq:dl-da}
  \acs{DL}(z) = (1+z)^2 \acs{DA}(z).
\end{equation}


%% EOF
